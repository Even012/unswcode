\documentclass{article}
\usepackage[utf8]{inputenc}
\usepackage{setspace}

\usepackage{geometry}
\geometry{a4paper,scale=0.80}

\title{Number Theory (P)}
\author{}
\date{}

\begin{document}
	\maketitle
	\begin{spacing}{1.5}
		
	\section*{Prove:}
	Prove that if $m = _{(_n)}m'$ and $p = _{(_n)}p'$ then $m+p = _{(_n)}m'+p'$
	
	\subsection*{proof}
	By the definition of mod, $ m = _{(_n)}p$ if $ n|(m-p) $. So we can get that $ n|(m-m') $ and $ n|(p-p') $.\\
	Therefore, for some $ k_1, k_2 \in Z$,
	\begin{center}
		$m=k_1n+m'$,\\
		$p=k_2n+p'$.
	\end{center}
	Add two equation, we can get that $m+p=(k_1+k_2)n+(m'+p')$. \\
	Because $ k_1, k_2 \in Z$, then $ k_1+k_2 \in Z $, so $ n|((m+p)-(m'+p'))$.\\
	Therefore $m+p=_{(_n)}(m'+p')$.
	
	\end{spacing}
\end{document}